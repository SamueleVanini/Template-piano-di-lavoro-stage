%----------------------------------------------------------------------------------------
%   USEFUL COMMANDS
%----------------------------------------------------------------------------------------

\newcommand{\dipartimento}{Dipartimento di Matematica ``Tullio Levi-Civita''}

%----------------------------------------------------------------------------------------
% 	USER DATA
%----------------------------------------------------------------------------------------

% Data di approvazione del piano da parte del tutor interno; nel formato GG Mese AAAA
% compilare inserendo al posto di GG 2 cifre per il giorno, e al posto di 
% AAAA 4 cifre per l'anno
\newcommand{\dataApprovazione}{24 Giugno 2021}

% Dati dello Studente
\newcommand{\nomeStudente}{Samuele}
\newcommand{\cognomeStudente}{Vanini}
\newcommand{\matricolaStudente}{1193535}
\newcommand{\emailStudente}{samuele.vanini@studenti.unipd.it}
\newcommand{\telStudente}{+ 39 346 82 48 863}

% Dati del Tutor Aziendale
\newcommand{\nomeTutorAziendale}{Dario}
\newcommand{\cognomeTutorAziendale}{Cescon}
\newcommand{\emailTutorAziendale}{dario.cescon@aton.com}
%\newcommand{\telTutorAziendale}{+ 39 000 00 00 000}
\newcommand{\ruoloTutorAziendale}{}

% Dati dell'Azienda
\newcommand{\ragioneSocAzienda}{Aton S.p.A}
\newcommand{\indirizzoAzienda}{Via Alessandro Volta 2, Lancenigo (TV)}
\newcommand{\sitoAzienda}{https://www.aton.eu/}
\newcommand{\emailAzienda}{mail@mail.it}
\newcommand{\partitaIVAAzienda}{P.IVA 02479320265}

% Dati del Tutor Interno (Docente)
\newcommand{\titoloTutorInterno}{Prof.}
\newcommand{\nomeTutorInterno}{Tullio}
\newcommand{\cognomeTutorInterno}{Vardanega}

\newcommand{\prospettoSettimanale}{
     % Personalizzare indicando in lista, i vari task settimana per settimana
     % sostituire a XX il totale ore della settimana
    \begin{itemize}
        \item \textbf{Prima Settimana (40 ore)}
        \begin{itemize}
            \item Incontro con persone coinvolte nel progetto per discutere i requisiti e le richieste
            relativamente al sistema da sviluppare;
            \item Verifica credenziali e strumenti di lavoro assegnati;
            \item Presa visione dell’infrastruttura esistente;
            \item Formazione sulle tecnologie adottate:
            \begin{itemize}
                \item strumenti;
                \item linguaggi;
                \item tools;
            \end{itemize}
            \item Formazione sugli apparati hardware adottati;
        \end{itemize}
        \item \textbf{Seconda Settimana - Sottotitolo (40 ore)} 
        \begin{itemize}
            \item Visione di progetti in essere con tecnologie simili;
            \item Studio autonomo riguardante:
            \begin{itemize}
                \item teoria RFID;
                \item archietettura software di un wrapper;
                \item documentazione e librerie del produttore Kathrein;
            \end{itemize}
        \end{itemize}
        \item \textbf{Terza Settimana - Sottotitolo (40 ore)} 
        \begin{itemize}
            \item Progettazione del modulo software;
            \item Definizione di una suite di test;
            \item Implementazione delle componenti riguardanti:
            \begin{itemize}
                \item lettura RFID;
                \item invio di dati ad un target esterno al device;
            \end{itemize}
        \end{itemize}
        \item \textbf{Quarta Settimana - Sottotitolo (40 ore)} 
        \begin{itemize}
            \item Implementazione delle componenti riguardanti:
            \begin{itemize}
                \item sessions;
                \item singulation;
            \end{itemize}
        \end{itemize}
        \item \textbf{Quinta Settimana - Sottotitolo (40 ore)} 
        \begin{itemize}
            \item Implementazione delle componenti riguardanti:
            \begin{itemize}
                \item transit time;
                \item dwell time;
            \end{itemize}
        \end{itemize}
        \newpage
        \item \textbf{Sesta Settimana - Sottotitolo (40 ore)} 
        \begin{itemize}
            \item Implementazione delle componenti riguardanti:
            \begin{itemize}
                \item kray protocol;
                \item GPIO handle;
            \end{itemize}
        \end{itemize}
        \item \textbf{Settima Settimana - Sottotitolo (40 ore)} 
        \begin{itemize}
            \item Studio di fattibilità su:
            \begin{itemize}
                \item refresh del contesto in caso di modifiche alla configurazione;
                \item built-in web server;
            \end{itemize}
        \end{itemize}
        \item \textbf{Ottava Settimana - Conclusione (40 ore)} 
        \begin{itemize}
            \item Studio di fattibilità sull'esposione all'esterno dell'hardware management;
            \item Collaudo e verifica del lavoro prodotto;
        \end{itemize}
    \end{itemize}
}

% Indicare il totale complessivo (deve essere compreso tra le 300 e le 320 ore)
\newcommand{\totaleOre}{320}

\newcommand{\obiettiviObbligatori}{
	 \item \underline{\textit{O01}}: Dimostrare una piena conoscenza della teoria legata alla tecnologia RFID;
	 \item \underline{\textit{O02}}: Realizzazione di un modulo software pienamente funzionante che rispetti le caratteristiche desiderate 
     dagli stakeholder;
}

\newcommand{\obiettiviDesiderabili}{
	 \item \underline{\textit{D01}}: studio di fattibilità sul porting del modulo software citato in precedenza all'interno del reader IOT;
}

\newcommand{\obiettiviFacoltativi}{
	 \item \underline{\textit{F01}}: esposizione tramite API o simili dello stato fisico del device di lettura RFID (es. antennas load, 
     RAM occupation, CPU load);
}